
% resumo em português
\begin{resumo}[Resumo] 
Trata-se de um elemento obrigatório. O título da seção (Resumo) deve ser centralizado e em negrito. A primeira frase deve ser significativa e relacionada ao tema. Deve apresentar os aspectos mais relevantes do trabalho, como objetivos, metodologia, resultados e conclusões. Usa-se o verbo na voz ativa e na terceira pessoa do singular. Evitar símbolos, contrações, fórmulas, equações, diagramas e recomenda-se evitar citações. Conforme a NBR 6028, a extensão do resumo deve conter de 150 a 500 palavras, dispostas em parágrafo único sem recuo e com espaçamento 1,5 entre linhas. Em seguida, deve-se indicar as palavras-chave, que devem ser separadas entre si por ponto e finalizadas também por ponto. 

% \noindent %- o resumo deve ter apenas 1 parágrafo e sem recuo de texto na primeira linha, essa tag remove o recuo. Não pode haver quebra de linha.

 \vspace{\onelineskip}
    
 \noindent
 \textbf{Palavras-chaves}: Palavra 1. Palavra 2. Palavra 3.
\end{resumo}



% resumo em inglês
\begin{resumo}[Abstract]
\begin{otherlanguage*}{english}

 %\noindent
Trata-se de uma versão do resumo para idioma de divulgação internacional. É um elemento obrigatório. Deve seguir os mesmos padrões do resumo na língua vernácula, conforma a NBR 6028. 



   \vspace{\onelineskip} 
 
   \noindent 
   \textbf{Keywords}: Palavra 1. Palavra 2. Palavra 3.
 \end{otherlanguage*}
 \end{resumo}
